\onecolumn{

\begin{figure}[H]
\begin{minipage}{0.25\linewidth}
\includegraphics[width=\linewidth]{image/logo/logo_DFA.jpg}
\end{minipage}
\hfill
\begin{minipage}{0.35\linewidth}
\includegraphics[width=\textwidth]{image/logo/logo_800anni.png}
\end{minipage}
\end{figure}

\noindent\makebox[\linewidth]{\color{linescolor} \rule[-0.2cm]{0.85\paperwidth}{1pt}}
\noindent\makebox[\linewidth]{\color{linescolor} \rule[0.3cm]{0.85\paperwidth}{1.2 pt}}


    
    \huge{\noindent\textbf{Nanofabrication, Characterization and \\
    Modelling of Colloidal Au Nanoparticles}}
    \vspace{4mm}
    
    \large{\textbf{Andrea De Bei, Tancredi Lo Presti Piccolo, Giovanni Piccolo}}
    \vspace{2mm}
    
    \footnotesize{Dipartimento di Fisica e Astronomia 'G. Galilei' - Università degli Studi di Padova} 
    \vspace{1mm}
    
    \footnotesize{\textit{Introduction to Nanophysics} course, a.y. 2020/2021}
    \vspace{1mm}
    
    \today

\vspace{4mm}


\small

\vspace{11pt}

\centerline{\rule{0.95\textwidth}{0.4pt}}

\begin{center}
    
    \begin{minipage}{0.9\textwidth}
    {\color{graytext}
        
        \noindent \textbf{Abstract:} In this paper we present the characterization and the modelling of spherical colloidal gold nanoparticles synthesized by means of the Turkevich method. The first part of this work starts with the description of the synthesis of the gold nanoparticles, followed by the measurement of the optical absorbance in the visible and near infra-red range (Vis-NIR spectroscopy). In order to get information about the size and concentration of the nanoparticles and about the refractive index of the surrounding medium a simulation by means of the Mie theory in dipolar approximation is performed.
        Consequently, a Grazing-incidence X-ray Diffraction (XRD) was performed on the nanoparticles deposited on a Si substrate in order to obtain measurements on the size of the particles and about their structure.
        In the end, a Scanning Electron Microscope was used in order to perform a morphological composition analysis of the Au nanoparticles by directly measuring them.
        
        \vspace{4mm}
        
        \noindent \textbf{Key words:} \textit{Au Nanoparticles, Mie Theory, Optical Characterization, X-Ray Diffraction, SEM Analysis}.
        }
    
    \end{minipage}
    
\end{center}

\vspace{2mm}

\centerline{\rule{0.95\textwidth}{0.4pt}}

\vspace{15pt}
}
